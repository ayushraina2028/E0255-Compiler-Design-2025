\documentclass{article}

\usepackage{listings} % For code formatting
\usepackage[utf8]{inputenc}  % For encoding support
\usepackage{amsmath}         % For mathematical formatting
\usepackage{graphicx}        % For including images
\usepackage{xcolor}
\usepackage[a4paper, left=0.5in, right=0.5in, top=0.5in, bottom=0.5in]{geometry}  % Adjust margins here
\usepackage{tcolorbox}
\usepackage{inconsolata}  % Use Inconsolata font (or replace with your choice)

% Define colors
\definecolor{codebg}{RGB}{240, 240, 240}  % Light gray background
\definecolor{framecolor}{RGB}{100, 100, 100}  % Dark gray frame
\definecolor{titlebg}{RGB}{30, 30, 30}  % Dark title background
\definecolor{titlefg}{RGB}{255, 255, 255}  % White title text

% Custom lstset
\lstset{
    language=C++,                    
    basicstyle=\ttfamily\footnotesize\fontfamily{zi4}\selectfont, % Use Inconsolata
    keywordstyle=\bfseries\color{blue},        
    commentstyle=\itshape\color{gray},        
    stringstyle=\color{red},          
    numbers=left,                     
    numberstyle=\tiny\color{blue},    
    frame=single,                     
    breaklines=true,                   
    captionpos=b,                      
    backgroundcolor=\color{codebg},  % Light gray background
    rulecolor=\color{framecolor},    % Dark frame
    tabsize=4                         
}

% Custom command to add a styled heading
\newtcbox{\codebox}{colback=titlebg, colframe=titlebg, colupper=titlefg, 
  boxrule=0pt, arc=5pt, left=5pt, right=5pt, top=3pt, bottom=3pt}

\title{Parallel Programming Notes}
\author{Ayush Raina}
\date{\today}

\begin{document} 

\maketitle

\subsection*{Sequential Fast Fourier Transform}
Given an input array $X[1:n]$, we need to output the transformed array $Y[1:n]$ where $Y[i] = \sum_{k=0}^{n-1} X[k] \omega^{ik}$ and $0 \leq i \leq n$ where $\omega^{ik}$ is called the twiddle factor and is the primitive $n^{th}$ root of unity. $\omega = e^{-2\pi \sqrt{-1}/n}$ and $i = \sqrt{-1}$ generally called as iota. \\

We can further simplify the computation as follows:
$Y[i] = \sum_{k=0}^{\frac{n}{2}-1} X[2k] \omega^{2ik} + \sum_{k=0}^{\frac{n}{2}-1} X[2k+1] \omega^{(2k+1)i}$ \\
$Y[i] = \sum_{k=0}^{\frac{n}{2}-1} X[2k] \omega^{2ik} + \omega^i \sum_{k=0}^{\frac{n}{2}-1} X[2k+1] \omega^{2ik}$ \\

Putting the value of $\omega = e^{-2\pi i/n}$, we get the final Expression as:

Add the remaining piece here

\newpage

\subsection*{Sequential FFT - Recursive Solution}
\begin{lstlisting}

procedure R_FFT(X,Y,n,w)
if(n == 1) then Y[0] = X[0]

begin
  Add this

\end{lstlisting}

\subsection*{Sequential FFT - Iterative Solution}
\begin{lstlisting}
  procedure I_FFT(X,Y,n)
  
\end{lstlisting}

\end{document}